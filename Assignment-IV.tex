% Options for packages loaded elsewhere
\PassOptionsToPackage{unicode}{hyperref}
\PassOptionsToPackage{hyphens}{url}
%
\documentclass[
]{article}
\usepackage{amsmath,amssymb}
\usepackage{lmodern}
\usepackage{iftex}
\ifPDFTeX
  \usepackage[T1]{fontenc}
  \usepackage[utf8]{inputenc}
  \usepackage{textcomp} % provide euro and other symbols
\else % if luatex or xetex
  \usepackage{unicode-math}
  \defaultfontfeatures{Scale=MatchLowercase}
  \defaultfontfeatures[\rmfamily]{Ligatures=TeX,Scale=1}
\fi
% Use upquote if available, for straight quotes in verbatim environments
\IfFileExists{upquote.sty}{\usepackage{upquote}}{}
\IfFileExists{microtype.sty}{% use microtype if available
  \usepackage[]{microtype}
  \UseMicrotypeSet[protrusion]{basicmath} % disable protrusion for tt fonts
}{}
\makeatletter
\@ifundefined{KOMAClassName}{% if non-KOMA class
  \IfFileExists{parskip.sty}{%
    \usepackage{parskip}
  }{% else
    \setlength{\parindent}{0pt}
    \setlength{\parskip}{6pt plus 2pt minus 1pt}}
}{% if KOMA class
  \KOMAoptions{parskip=half}}
\makeatother
\usepackage{xcolor}
\usepackage[margin=1in]{geometry}
\usepackage{color}
\usepackage{fancyvrb}
\newcommand{\VerbBar}{|}
\newcommand{\VERB}{\Verb[commandchars=\\\{\}]}
\DefineVerbatimEnvironment{Highlighting}{Verbatim}{commandchars=\\\{\}}
% Add ',fontsize=\small' for more characters per line
\usepackage{framed}
\definecolor{shadecolor}{RGB}{248,248,248}
\newenvironment{Shaded}{\begin{snugshade}}{\end{snugshade}}
\newcommand{\AlertTok}[1]{\textcolor[rgb]{0.94,0.16,0.16}{#1}}
\newcommand{\AnnotationTok}[1]{\textcolor[rgb]{0.56,0.35,0.01}{\textbf{\textit{#1}}}}
\newcommand{\AttributeTok}[1]{\textcolor[rgb]{0.77,0.63,0.00}{#1}}
\newcommand{\BaseNTok}[1]{\textcolor[rgb]{0.00,0.00,0.81}{#1}}
\newcommand{\BuiltInTok}[1]{#1}
\newcommand{\CharTok}[1]{\textcolor[rgb]{0.31,0.60,0.02}{#1}}
\newcommand{\CommentTok}[1]{\textcolor[rgb]{0.56,0.35,0.01}{\textit{#1}}}
\newcommand{\CommentVarTok}[1]{\textcolor[rgb]{0.56,0.35,0.01}{\textbf{\textit{#1}}}}
\newcommand{\ConstantTok}[1]{\textcolor[rgb]{0.00,0.00,0.00}{#1}}
\newcommand{\ControlFlowTok}[1]{\textcolor[rgb]{0.13,0.29,0.53}{\textbf{#1}}}
\newcommand{\DataTypeTok}[1]{\textcolor[rgb]{0.13,0.29,0.53}{#1}}
\newcommand{\DecValTok}[1]{\textcolor[rgb]{0.00,0.00,0.81}{#1}}
\newcommand{\DocumentationTok}[1]{\textcolor[rgb]{0.56,0.35,0.01}{\textbf{\textit{#1}}}}
\newcommand{\ErrorTok}[1]{\textcolor[rgb]{0.64,0.00,0.00}{\textbf{#1}}}
\newcommand{\ExtensionTok}[1]{#1}
\newcommand{\FloatTok}[1]{\textcolor[rgb]{0.00,0.00,0.81}{#1}}
\newcommand{\FunctionTok}[1]{\textcolor[rgb]{0.00,0.00,0.00}{#1}}
\newcommand{\ImportTok}[1]{#1}
\newcommand{\InformationTok}[1]{\textcolor[rgb]{0.56,0.35,0.01}{\textbf{\textit{#1}}}}
\newcommand{\KeywordTok}[1]{\textcolor[rgb]{0.13,0.29,0.53}{\textbf{#1}}}
\newcommand{\NormalTok}[1]{#1}
\newcommand{\OperatorTok}[1]{\textcolor[rgb]{0.81,0.36,0.00}{\textbf{#1}}}
\newcommand{\OtherTok}[1]{\textcolor[rgb]{0.56,0.35,0.01}{#1}}
\newcommand{\PreprocessorTok}[1]{\textcolor[rgb]{0.56,0.35,0.01}{\textit{#1}}}
\newcommand{\RegionMarkerTok}[1]{#1}
\newcommand{\SpecialCharTok}[1]{\textcolor[rgb]{0.00,0.00,0.00}{#1}}
\newcommand{\SpecialStringTok}[1]{\textcolor[rgb]{0.31,0.60,0.02}{#1}}
\newcommand{\StringTok}[1]{\textcolor[rgb]{0.31,0.60,0.02}{#1}}
\newcommand{\VariableTok}[1]{\textcolor[rgb]{0.00,0.00,0.00}{#1}}
\newcommand{\VerbatimStringTok}[1]{\textcolor[rgb]{0.31,0.60,0.02}{#1}}
\newcommand{\WarningTok}[1]{\textcolor[rgb]{0.56,0.35,0.01}{\textbf{\textit{#1}}}}
\usepackage{graphicx}
\makeatletter
\def\maxwidth{\ifdim\Gin@nat@width>\linewidth\linewidth\else\Gin@nat@width\fi}
\def\maxheight{\ifdim\Gin@nat@height>\textheight\textheight\else\Gin@nat@height\fi}
\makeatother
% Scale images if necessary, so that they will not overflow the page
% margins by default, and it is still possible to overwrite the defaults
% using explicit options in \includegraphics[width, height, ...]{}
\setkeys{Gin}{width=\maxwidth,height=\maxheight,keepaspectratio}
% Set default figure placement to htbp
\makeatletter
\def\fps@figure{htbp}
\makeatother
\setlength{\emergencystretch}{3em} % prevent overfull lines
\providecommand{\tightlist}{%
  \setlength{\itemsep}{0pt}\setlength{\parskip}{0pt}}
\setcounter{secnumdepth}{-\maxdimen} % remove section numbering
\ifLuaTeX
  \usepackage{selnolig}  % disable illegal ligatures
\fi
\IfFileExists{bookmark.sty}{\usepackage{bookmark}}{\usepackage{hyperref}}
\IfFileExists{xurl.sty}{\usepackage{xurl}}{} % add URL line breaks if available
\urlstyle{same} % disable monospaced font for URLs
\hypersetup{
  pdftitle={Assignment IV},
  pdfauthor={Jeremy Gachanja},
  hidelinks,
  pdfcreator={LaTeX via pandoc}}

\title{Assignment IV}
\author{Jeremy Gachanja}
\date{8/7/2022}

\begin{document}
\maketitle

\hypertarget{b-collins-bought-a-share-of-stock-for-12-and-it-is-believed-that-the-stock-price-moves-day-by-day-as-a-simple-random-walk-with-p-0.58.-what-is-the-probability-that-collins-stock-reaches-the-high-value-of-35-before-the-low-value-of-8}{%
\section{b) Collins bought a share of stock for \$12, and it is believed
that the stock price moves (day by day) as a simple random walk with p =
0.58. What is the probability that Collins' stock reaches the high value
of \$35 before the low value of
\$8?}\label{b-collins-bought-a-share-of-stock-for-12-and-it-is-believed-that-the-stock-price-moves-day-by-day-as-a-simple-random-walk-with-p-0.58.-what-is-the-probability-that-collins-stock-reaches-the-high-value-of-35-before-the-low-value-of-8}}

\hypertarget{solution}{%
\subsection{solution}\label{solution}}

\$\frac{1- (\frac{q}{p})^b}{1- (\frac{q}{p})^{a+b}} \$

where a is the up movement and b is the down movement

\begin{Shaded}
\begin{Highlighting}[]
\NormalTok{origin }\OtherTok{=} \DecValTok{12}
\NormalTok{p }\OtherTok{=} \FloatTok{0.58}
\NormalTok{q }\OtherTok{=} \DecValTok{1}\SpecialCharTok{{-}}\NormalTok{p}
\NormalTok{up }\OtherTok{=} \DecValTok{35}
\NormalTok{down }\OtherTok{=} \DecValTok{8}
\NormalTok{a }\OtherTok{=}\NormalTok{ up}\SpecialCharTok{{-}}\NormalTok{origin}
\NormalTok{b }\OtherTok{=}\NormalTok{ origin}\SpecialCharTok{{-}}\NormalTok{down}
\NormalTok{p\_35 }\OtherTok{=}\NormalTok{ (}\DecValTok{1}\SpecialCharTok{{-}}\NormalTok{(q}\SpecialCharTok{/}\NormalTok{p)}\SpecialCharTok{\^{}}\NormalTok{b)}\SpecialCharTok{/}\NormalTok{(}\DecValTok{1}\SpecialCharTok{{-}}\NormalTok{(q}\SpecialCharTok{/}\NormalTok{p)}\SpecialCharTok{\^{}}\NormalTok{(a}\SpecialCharTok{+}\NormalTok{b))}
\FunctionTok{print}\NormalTok{(p\_35)}
\end{Highlighting}
\end{Shaded}

\begin{verbatim}
## [1] 0.7251491
\end{verbatim}

\hypertarget{c-explain-clearly-the-difference-between-the-following-terms-as-used-in-markov-chains}{%
\section{c) Explain clearly the difference between the following terms
as used in Markov
Chains}\label{c-explain-clearly-the-difference-between-the-following-terms-as-used-in-markov-chains}}

\hypertarget{i.-communicating-class-and-absorption-state}{%
\subsection{i. Communicating class and absorption
state}\label{i.-communicating-class-and-absorption-state}}

An absorption state i is one which is impossible to leave.

Points i and j in a markov chain are said to communicate if i is
reachable to j and j is rachable to i. This makes the markov chain
irreducible and thus a communicating class.

\hypertarget{ii.-recurrence-and-nonrecurrence-state}{%
\subsection{ii. Recurrence and nonrecurrence
state}\label{ii.-recurrence-and-nonrecurrence-state}}

A recurrent state i is recurrent if and only if one starts at state i
and there is a probability of 1 that one will eventually return to state
i.

A non recurrent or transient state is one where if one starts at a state
i the probability of returning to that state i is less than 1

\hypertarget{iii.-periodicity-and-aperiodic}{%
\subsection{iii. Periodicity and
aperiodic}\label{iii.-periodicity-and-aperiodic}}

A state i is periodic if one starts at state i and the number of steps
to get back to that step i is greater than one.

A state i is aperiodic if one starts at state i and the number of steps
to get back to that step i is equal to one.

\hypertarget{iv.-ergodic-chain-and-transient-state}{%
\subsection{iv. Ergodic chain and transient
state}\label{iv.-ergodic-chain-and-transient-state}}

An egordic chain is one that satisfies three conditions that is the
chain is aperiodic, recurrent and irreducible.

A transient state is one where if one starts at a state i the
probability of returning to that state i is less than 1.

\hypertarget{v.-reducible-and-irreducible}{%
\subsection{v. Reducible and
irreducible}\label{v.-reducible-and-irreducible}}

An irreducible chain is one where it is possible to move from any state
to any other state regardless of if the path is direct or indirect.

An irreducible chain is one where it is not possible to move from any
state to any other state.

\hypertarget{e-clearly-specify-five-components-of-a-hidden-markov-model}{%
\section{e) Clearly specify five components of a Hidden Markov
Model}\label{e-clearly-specify-five-components-of-a-hidden-markov-model}}

\hypertarget{f-use-chapman-kolmogorov-postulates-to-derive-the-poisson-process.-also-derive-the-mean-and-variance-of-the-poisson-process.}{%
\section{f) Use Chapman Kolmogorov postulates to derive the Poisson
Process. Also derive the mean and variance of the Poisson
process.}\label{f-use-chapman-kolmogorov-postulates-to-derive-the-poisson-process.-also-derive-the-mean-and-variance-of-the-poisson-process.}}

\hypertarget{g-a-certain-stock-price-has-been-observed-to-follow-a-pattern.-if-the-stock-price-goes-up-one-day-theres-a-25-chance-of-it-rising-tomorrow-a-35-chance-of-it-falling-and-a-40-chance-of-it-remaining-the-same.-if-the-stock-price-falls-one-day-theres-a-25-chance-of-it-rising-tomorrow-a-50-chance-of-it-falling-and-a-25-chance-of-it-remaining-the-same.-finally-if-the-price-is-stable-on-one-day-then-it-has-a-50-50-change-of-rising-or-falling-the-next-day.}{%
\section{g) A certain stock price has been observed to follow a pattern.
If the stock price goes up one day, there's a 25\% chance of it rising
tomorrow, a 35\% chance of it falling, and a 40\% chance of it remaining
the same. If the stock price falls one day, there's a 25\% chance of it
rising tomorrow, a 50\% chance of it falling, and a 25\% chance of it
remaining the same. Finally, if the price is stable on one day, then it
has a 50-50 change of rising or falling the next
day.}\label{g-a-certain-stock-price-has-been-observed-to-follow-a-pattern.-if-the-stock-price-goes-up-one-day-theres-a-25-chance-of-it-rising-tomorrow-a-35-chance-of-it-falling-and-a-40-chance-of-it-remaining-the-same.-if-the-stock-price-falls-one-day-theres-a-25-chance-of-it-rising-tomorrow-a-50-chance-of-it-falling-and-a-25-chance-of-it-remaining-the-same.-finally-if-the-price-is-stable-on-one-day-then-it-has-a-50-50-change-of-rising-or-falling-the-next-day.}}

\hypertarget{i.-generate-the-transition-matrix}{%
\subsection{i. Generate the transition
matrix}\label{i.-generate-the-transition-matrix}}

\hypertarget{solution-1}{%
\subsubsection{solution}\label{solution-1}}

\begin{Shaded}
\begin{Highlighting}[]
\FunctionTok{library}\NormalTok{(markovchain)}
\end{Highlighting}
\end{Shaded}

\begin{verbatim}
## Package:  markovchain
## Version:  0.9.0
## Date:     2022-07-01
## BugReport: https://github.com/spedygiorgio/markovchain/issues
\end{verbatim}

\begin{Shaded}
\begin{Highlighting}[]
\NormalTok{transition\_mat}\OtherTok{=}\FunctionTok{matrix}\NormalTok{(}\AttributeTok{nrow =} \DecValTok{3}\NormalTok{,}\AttributeTok{ncol=}\DecValTok{3}\NormalTok{,}\FunctionTok{c}\NormalTok{(}\FloatTok{0.25}\NormalTok{,}\FloatTok{0.25}\NormalTok{,}\FloatTok{0.5}\NormalTok{,}\FloatTok{0.35}\NormalTok{,}\FloatTok{0.5}\NormalTok{,}\FloatTok{0.5}\NormalTok{,}\FloatTok{0.4}\NormalTok{,}\FloatTok{0.25}\NormalTok{,}\DecValTok{0}\NormalTok{))}
\NormalTok{statesNames }\OtherTok{\textless{}{-}} \FunctionTok{c}\NormalTok{(}\StringTok{"Up"}\NormalTok{, }\StringTok{"Down"}\NormalTok{, }\StringTok{"Stable"}\NormalTok{)}
\NormalTok{markovB }\OtherTok{\textless{}{-}} \FunctionTok{new}\NormalTok{(}\StringTok{"markovchain"}\NormalTok{, }\AttributeTok{states =}\NormalTok{ statesNames, }\AttributeTok{transitionMatrix =}\NormalTok{transition\_mat, }\AttributeTok{name =} \StringTok{"A markovchain Object"}\NormalTok{)}
\FunctionTok{print}\NormalTok{(markovB)}
\end{Highlighting}
\end{Shaded}

\begin{verbatim}
##          Up Down Stable
## Up     0.25 0.35   0.40
## Down   0.25 0.50   0.25
## Stable 0.50 0.50   0.00
\end{verbatim}

\hypertarget{ii.-draw-the-markov-chain-using-r}{%
\subsection{ii. Draw the Markov chain using
R}\label{ii.-draw-the-markov-chain-using-r}}

\hypertarget{solution-2}{%
\subsubsection{solution}\label{solution-2}}

\begin{Shaded}
\begin{Highlighting}[]
\FunctionTok{library}\NormalTok{(diagram)}
\end{Highlighting}
\end{Shaded}

\begin{verbatim}
## Loading required package: shape
\end{verbatim}

\begin{Shaded}
\begin{Highlighting}[]
\FunctionTok{rownames}\NormalTok{(transition\_mat) }\OtherTok{=}\NormalTok{ statesNames}
\FunctionTok{colnames}\NormalTok{(transition\_mat) }\OtherTok{=}\NormalTok{ statesNames}
\FunctionTok{plotmat}\NormalTok{(transition\_mat,}\AttributeTok{relsize =}\NormalTok{ .}\DecValTok{65}\NormalTok{,}\AttributeTok{cex=}\FloatTok{0.7}\NormalTok{)}
\end{Highlighting}
\end{Shaded}

\includegraphics{Assignment-IV_files/figure-latex/unnamed-chunk-3-1.pdf}

\hypertarget{iii.-determine-if-the-chain-is-ergodic}{%
\subsection{iii. Determine if the chain is
Ergodic}\label{iii.-determine-if-the-chain-is-ergodic}}

\hypertarget{solution-3}{%
\subsubsection{solution}\label{solution-3}}

The chain is ergodic since it is recurrent, aperiodic and irreducible.

\hypertarget{iv.-find-the-limiting-distribution-of-the-transition-matrix}{%
\subsection{iv. Find the limiting distribution of the transition
matrix}\label{iv.-find-the-limiting-distribution-of-the-transition-matrix}}

\hypertarget{solution-4}{%
\subsubsection{solution}\label{solution-4}}

\begin{Shaded}
\begin{Highlighting}[]
\FunctionTok{steadyStates}\NormalTok{(markovB)}
\end{Highlighting}
\end{Shaded}

\begin{verbatim}
##             Up      Down    Stable
## [1,] 0.3092784 0.4536082 0.2371134
\end{verbatim}

\hypertarget{h-a-telephone-attendant-receives-110-calls-during-the-busy-hour.-each-call-takes-on-average-2.1-minutes-to-process.}{%
\section{h) A telephone attendant receives 110 calls during the busy
hour. Each call takes, on average, 2.1 minutes to
process.}\label{h-a-telephone-attendant-receives-110-calls-during-the-busy-hour.-each-call-takes-on-average-2.1-minutes-to-process.}}

\hypertarget{a-what-percentage-of-the-attendants-time-is-devoted-to-answering-calls}{%
\subsection{a) What percentage of the attendant's time is devoted to
answering
calls?}\label{a-what-percentage-of-the-attendants-time-is-devoted-to-answering-calls}}

\hypertarget{b-how-long-must-people-wait-on-average-before-their-call-is-processed}{%
\subsection{b) How long must people wait, on average, before their call
is
processed?}\label{b-how-long-must-people-wait-on-average-before-their-call-is-processed}}

\end{document}
